\section{Introduction and Physics Motivation}

The {\bf{Micro}} {\bf{Boo}}ster {\bf{N}}eutrino {\bf{E}}xperiment (MicroBooNE) employes a large ($\sim$100 tonnes) Liquid Argon Time Projection Chamber (LArTPC) detector designed for precision neutrino physics measurements.   MicroBooNE is the latest among a family of detectors that exploit the potential of liquified noble gases as the detection medium for neutrino interactions.   These detectors combine the advantages of high spatial resolution and calorimetry for excellent particle identification with the potential to scale to very large volumes. 

Large calorimeters using cryogenic noble liquids combined with active components were recognized in the 1970s as having use for particle physics applications~\cite{Willis:1974}.  Specifically, much of the liquid argon based technology was developed within the ICARUS program ~\cite{Benetti:1993-3ton,Cennini:1994-3ton,Arneodo:1999-50l} culminating in the realization of the ICARUS T600 detector~\cite{Amerio:2004-T600}.  On a much smaller scale than the ICARUS detector, the ArgoNeuT (Argon Neutrino Test) experiment operated a $\sim$0.25 tonne \lartpc from 2009-2010 in the NuMI neutrino beam at Fermilab.   ArgoNeuT performed a series of detailed studies on the interaction of medium-energy neutrinos \cite{Acciarri:2013-argoneut-recomb}  producing the first published neutrino cross section measurements on argon~\cite{Anderson:2012-argoneut-CCincl,Acciarri:2014-argoneut-CCxsec,Acciarri:2014eit}.   Next generation \lartpcs for the Short Baseline Neutrino Detector (SBND) experiment and the Deep Underground Neutrino Experiment (DUNE) are now being designed and constructed.   
   
MicroBooNE 's principal physics goal is to address short baseline neutrino oscillations, primarily the MiniBooNE observation of an excess of electron-like events at low energy~\cite{AguilarArevalo:2008rc}, at the Fermi National Accelerator Laboratory (Fermilab).  MicroBooNE will be exposed to the  0.5-2 GeV on-axis Booster Neutrino Beam (BNB) at a $\sim$500~m baseline, the same as was employed for MiniBoonE.  The MicroBooNE experiment is exploiting the \lartpc technology because of its superior capability for separation of signal electrons from the background of photon conversions.   While the mass of MicroBooNE is significantly less than the mass of MiniBooNE, this superior discrimination is expected to address the MiniBooNE result at the 5$\sigma$ level.   

%This is the same beam as employed for MiniBooNE, but MicroBooNE will take advantage of the \lartpc technology to efficiently separate signal electrons from background photons. 

In addition to MicroBooNE's signature oscillation analyses, a suite of precision cross-section measurements will be performed, critical both for future \lartpc oscillation experiments and for understanding neutrino interactions in general.   In the BNB, multiple interaction processes (quasi-elastic, resonances, deep inelastic scattering) are possible, and complicated nuclear effects in neutrino interactions on argon result in a variety of final states. These can range from the emission of several nucleons to more complex topologies with multiple pions, all in addition to the leading lepton in charged-current events. The \lartpc technology employed by MicroBooNE is particularly well suited for complicated topologies because of its excellent particle identification capability and calorimetric energy reconstruction down to very low detection thresholds. MicroBooNE's physics program also encompasses searches for supernova and proton decay.  The detector is capable of recording neutrinos from a galactic supernova which would result in $\sim$30 charged current neutrino interactions in MicroBooNE's active volume.   The detector will measure proton decay-like signatures and backgrounds and develop the analysis for this search; though its target mass is insufficiently large to enable a competitive sensitivity, the analysis will provide an important proof-of-principle for future searches in more massive detectors.  

%MicroBooNE is located 470~m from the BNB production target and 600~m from the NuMI production target.  

MicroBooNE began operations in late 2015 for an initial anticipated $\sim$3 year run.   In 2018, MicroBooNE will continue operations as part of an expanded Short Baseline Neutrino (SBN) program~\cite{Adams:2013-lar1nd} at Fermilab that includes continued operation of MicroBooNE (at 470~m) along with the SBND (at 110~m)  and ICARUS (at 600~m) detectors.  The SBND and ICARUS experimental halls and detectors are presently under construction.  MicroBooNE will definitively address whether or not the MiniBooNE low energy excess in neutrino mode is due to electrons or photons in its initial run.  SBND will look for this low energy excess at the near location and ICARUS, with its larger mass, will enable the three detector program to cover the entire LSND-allowed region in neutrino parameter space with 5$\sigma$ sensitivity in the $\nu_e$ appearance channel.


 
% \paragraph{Organization of Document}

This document describes the design, construction, and technical details of the MicroBooNE detector.  Section \ref{sec:overview} gives a brief review of the \lartpc technique and its implementation in MicroBooNE.  Section~\ref{sec:cryostat} describes the cryogenic and purification systems which are required for maintaining a stable volume of highly purified liquid argon.  The \lartpc described in section~\ref{sec:tpc-all} is the centerpiece of the experiment, providing fine-grained images of neutrino interactions.  A light collection system, described in section~\ref{sec:light-collection}, provides timing information, used primarily for triggering beam events, from the prompt scintillation light that is produced in the detector volume. Signals from the light collection system and from the \lartpc are amplified, sampled, and recorded by a custom-designed electronic and readout system, as described in section~\ref{sec:electronics}.  Section \ref{sec:slow-control} describes the auxiliary instrumentation that monitor and control the detector and all of its associated systems, as well as provide an electrically quiet environment for the experiment to operate. Finally, one of the main calibration sources for the experiment is an ultraviolet laser system, described in section~\ref{sec:laser}, that provides the capability to map out geometric track distortions, as induced, for example, by space charge.  A cosmic ray tagger system, under construction at the time of the writing of this paper, will surround the detector to improve cosmic ray identification and rejection.  This system will be described in a subsequent publication.  

More information on the \lartpc technology can be found in existing reviews (see, e.g.,~\cite{Marchionni:2013} and references therein).
