\section{Introduction and Physics Motivation}
%{\it{Antonio}} & BTF

Neutrino detection is entering a new era with a number of large precision neutrino detectors, specifically Liquid Argon Time Projection Chambers (LArTPCs), now in design, construction, data taking, and analysis, both for long and short baseline neutrino oscillation experiments.  The {\bf{Micro}} {\bf{Boo}}ster {\bf{N}}eutrino {\bf{E}}xperiment, henceforth referred to as MicroBooNE, is the first large ($\sim$100 tons) \lartpc to operate in the United States and the second only after ICARUS to operate worldwide.  \lartpc detectors combine fine-grained tracking with total absorption calorimetry to provide excellent signal efficiency and background rejection.  

The MicroBooNE experiment combines a physics program of short-baseline oscillations and neutrino cross-section measurements with development goals to inform larger scale construction of \lartpcs for the long-baseline program.  MicroBooNE's principal physics goal is to address the long standing unsolved puzzles generated by short-baseline neutrino oscillation results from past experiments. These include results from accelerator based experiments, LSND and MiniBooNE, reactor experiments, and radiochemical experiments.  The LSND accelerator based oscillation experiment observed  anti-electron neutrino appearance in an anti-muon neutrino beam.  The MiniBooNE experiment \cite{AguilarArevalo:2008rc} also investigated (anti)muon- to (anti)electron-neutrino oscillations and found a significant deviation from the expectation in the number of electron (anti)neutrino candidates at low energy ($\sim$ 200-500 MeV).  Other hints come from a re-analysis of the expected reactor antineutrino flux that suggests a deficit of reactor neutrinos \cite{reactor anomaly}, and the analysis of calibration data taken by GALLEX and SAGE \cite{gallexSage} radiochemical experiments.   Interpreting these results through neutrino oscillations suggests a mass squared difference on the order of 1$\sim eV^2$ and an oscillation probability consistent with small mixing, incompatible with the oscillation parameters from solar and atmospheric results \cite{solar}.  These signals can be interpreted through oscillations to sterile neutrinos ie: a neutrino which does not interact via the weak interaction but which can oscillate with the three standard model neutrinos.  While each of the short baseline results taken alone does not reach the statistical significance to claim a discovery, combining them together suggests sterile neutrino oscillations or some other phenomena underway at short baselines.   Definitive evidence for sterile neutrinos would constitute a revolutionary discovery, with strong implications for particle physics as well as for cosmology. Proposals to address these signals by employing reactor, accelerator, and radioactive source experiments are in the planning stages or underway worldwide.
   
MicroBooNE will address short baseline neutrino oscillations, specifically the MiniBooNE low energy excess result, at Fermi National Accelerator Laboratory (Fermilab).  MicroBooNE will be exposed to the  0.5-2 GeV on-axis Booster Neutrino beam, the same beam used for the MiniBooNE experiment, at a $\sim 500$m baseline, also the same as MiniBooNE, but by employing the \lartpc technology to efficiently separate signal electrons from background photons.  In addition to MicroBooNE's signature oscillation analyses, a suite of precision cross-section measurements will be performed, critical both for future \lartpc oscillation experiments and for what can be learned about neutrino interactions in general.   In the neutrino interactions from FNAL's BNB, multiple interaction processes (quasi-elastic, resonances, deep inelastic scattering) are possible, and complicated nuclear effects in neutrino interactions on argon result in a variety of final states. These can range from the emission of several nucleons to more complex topologies with multiple pions, all in addition to the leading lepton in charged-current events. The \lartpc technology employed by MicroBooNE is particularly well suited for complicated topologies because of its excellent particle identification capability and calorimetric energy reconstruction down to very low detection thresholds. MicroBooNE will also be supernova live and capable of detecting a galactic supernova, and prepare the analysis of  the search for proton decays. For a proton decay search, the experiment will not have a competitive sensitivity due to too small target mass. However, this will be an important proof of principle measurement towards future searches in larger detectors.  

MicroBooNE is the first phase of a staged program of neutrino detectors at short baseline (the Short Baseline Neutrino (SBN) program at Fermilab) exposed to the on-axis BNB beam and an off-axis component of the NuMI beam (Neutrinos from the Main Injector) at Fermilab.  

The MicroBooNE experiment is located 470~m from the BNB production target and 600~m from the NuMI production target.  MicroBooNE began operations in late 2015 for an anticipated $\sim$3 year data taking run to be followed by data taking as part of the full SBN program beginning in 2018 with the commencement of operations of the SBND \cite{send} and ICARUS experiments, located at 110m and 600m respectively from the BNB production target.  MicroBooNE will definitively address whether or not the MiniBooNE low energy excess in neutrino mode is due to electrons or photons.  SBND will look for this low energy excess at the near location (110m) from the production target, and the three detectors combined will cover the entire LSND allowed region in neutrino parameter space to 5$\sigma$ through $\nu_e$ appearance.

The \lartpc technique will be used by MicroBooNE at a relatively large scale, so far only surpassed by the ICARUS T600 detector. The potential of liquified noble gases to be employed as the detection media in particle detectors with high spatial resolution was recognized in the 1960's (see, e.g., Ref.~\cite{Doke:1993}). Large calorimeters for the measurement of particle energy were then realized by using cryogenic noble liquids as active components (see, e.g., Ref.~\cite{Willis:1974}). Furthermore, the \lartpc was proposed at CERN by C.~Rubbia in 1977~\cite{Rubbia:1977} as a detector capable of providing uniform and precise imaging of large volumes. 

The implementation of the \lartpc idea was extensively developed within the ICARUS experiment through an intense R\&D program. Various prototype detectors were realized and operated to develop the systems needed for larger physics experiments.  Techniques for purification, signal readout electronics, and cryogenics were developed. The largest ICARUS test device had a mass of about 3 tons of LAr~\cite{Benetti:1993-3ton,Cennini:1994-3ton} and allowed for the collection of large samples of cosmic-ray events. In parallel, a smaller volume detector (50 liters of LAr) was exposed to the CERN WANF neutrino beam allowing the detection of neutrino interaction events for the first time in a \lartpc~\cite{Arneodo:1999-50l}. 

The largest \lartpc detector built so far is the 600-ton ICARUS T600 detector, which was operated first on the surface exposed to cosmic rays~\cite{Amerio:2004-T600} before deployment underground at LNGS for operation in the CNGS neutrino beam.  ICARUS measurements have shown that the technique is viable at large scales with a drift length as long as 1.5~m~\cite{Amoruso:2004-muon,Amoruso:2004-electron,Antonello:2004-cherenkov,Amoruso:2004-purity,Arneodo:2003-tracks}. 

The success of the construction of the T600 motivated several follow up ideas aimed at reaching even higher target masses (see, e.g., Ref.~\cite{Aprili:2002-proposal}). A monolithic magnetized large mass device was proposed in~\cite{Cline:2003-LANNDD}, based on the scaling up of the ICARUS approach. The ICARUS experience led then to the broader development of the \lartpc concept. A \lartpc design envisioning a single cylindrical volume of 70~m diameter and 20~m height, called GLACIER, was discussed in~\cite{Rubbia:2004-glacier,Rubbia:2009-glacier}, based on a double-phase readout. This approach eventually evolved to the LAGUNA~\cite{LAGUNAcollab} proposal and to the LAGUNA LBNO~\cite{Stahl:2012-LBNO,Agarwalla:2013-LBNO} conceptual design, aimed at a large detector placed along a long-baseline beam from CERN to an European underground site.  In the United States, an analogous multi-kton detector was proposed using conventional single-phase wire readout for the LBNE project~\cite{Adams:2013-LBNE}, aimed at a long-baseline neutrino beam sent from Fermilab to the Sanford Laboratory in South Dakota.  The LBNE and LBNO efforts have now largely merged to form the DUNE experiment \cite{DUNE}. 

In parallel, \lartpcs were also considered for short-baseline neutrino experiments, motivating the use of relatively smaller size devices.  At Fermilab, the first \lartpc to be successfully operated in the NuMI beam was the ArgoNeuT detector~\cite{Anderson:2012-argoneut,Acciarri:2013-argoneut-recomb,Anderson:2012-argoneut-CCincl,Acciarri:2014-argoneut-CCxsec,Acciarri:2014-argoneut-CCcohpi} that, despite the relatively small mass of $\sim170$~kg, has been able to perform a series of detailed studies on the interaction of medium-energy neutrinos\cite{argoneutpapers}.  Other critical development projects are currently underway at Fermilab such as ~\cite{Cavanna:2014-lariat,Szelc:2013-lariatlight,Adamowski:2014-LAPD,Rebel:2011-MTS,Montanari:2013-35ton}, all geared towards development of the technology and of analyzing particle interactions in the detectors.   The goal of the development in the near term is towards the SBN program experiments, and in the farther term, following the recent recommendations of the P5 panel~\cite{P5:2014}, the LBNF project and the DUNE experiment which envisions an internationally coordinated effort for the realization of an intense neutrino beam facility at Fermilab and a far large mass neutrino detector in a long-baseline configuration ($>1000$~km). 

More information on the \lartpc technology can be found in existing reviews (see, e.g.,~\cite{Marchionni:2013} and references therein).
 
 \paragraph{Organization of Document}

This document describes the design, construction, and technical details of the MicroBooNE experiment.  Section \ref{sec:overview} gives a brief review of the \lartpc technique and its implementation in MicroBooNE.  Section~\ref{sec:cryostat} describes the cryogenic and purification systems which are required for maintaining a stable volume of highly purified liquid argon.  The \lartpc detector described in section~\ref{sec:tpc-all} is the centerpiece of the experiment, providing fine-grained images of neutrino interactions.  A light collection system, described in section~\ref{sec:light-collection}, provides precise timing information by viewing the detector volume and recording signals from the scintillation light that is produced therein. Signals from the light collection system and from the \lartpc are amplified, sampled, and recorded by a custom-designed electronic and readout system, as described in section~\ref{sec:electronics}.  Section \ref{sec:slow-control} describes the auxiliary instrumentation that monitor and control the detector and all of its associated systems, as well as provide an electrically quiet environment for the experiment to operate. Finally, one of the main calibration sources for the experiment is an ultraviolet laser system, described in section~\ref{sec:laser}, that provides the capability to map out the performance of the \lartpc detector.  A cosmic ray tagger system, under construction at the time of the writing of this paper, will be sited to beam right and left, and above and below the detector cryostat.

